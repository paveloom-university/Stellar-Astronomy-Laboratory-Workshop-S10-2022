\documentclass[a4paper, oneside]{article}
\special{pdf:minorversion 6}

\usepackage{geometry}
\geometry{
  textwidth=358.0pt,
  textheight=608.0pt,
  top=90pt,
  left=113pt,
}

\usepackage[english, russian]{babel}

\usepackage{fontspec}
\setmainfont[
  Ligatures=TeX,
  Extension=.otf,
  BoldFont=cmunbx,
  ItalicFont=cmunti,
  BoldItalicFont=cmunbi,
]{cmunrm}
\usepackage{unicode-math}

\usepackage[bookmarks=false]{hyperref}
\hypersetup{pdfstartview={FitH},
            colorlinks=true,
            linkcolor=magenta,
            pdfauthor={Павел Соболев}}

\usepackage{calrsfs}
\DeclareMathAlphabet{\pazocal}{OMS}{zplm}{m}{n}

\usepackage[table]{xcolor}
\usepackage{booktabs}
\usepackage{caption}

\usepackage{float}
\usepackage{subcaption}
\usepackage{graphicx}
\graphicspath{ {../plots/} }
\DeclareGraphicsExtensions{.pdf, .png}

\usepackage{sectsty}
\sectionfont{\centering}
\subsubsectionfont{\centering\normalfont\itshape}

\newcommand{\su}{\vspace{-0.5em}}
\newcommand{\npar}{\par\vspace{\baselineskip}}

\setlength{\parindent}{0pt}

\DeclareMathOperator{\atantwo}{atan2}

\usepackage{diagbox}

\newlength{\imagewidth}
\newlength{\imageheight}
\newcommand{\subgraphics}[1]{
\settowidth{\imagewidth}{\includegraphics[height=\imageheight]{#1}}%
\begin{subfigure}{\imagewidth}%
    \includegraphics[height=\imageheight]{#1}%
\end{subfigure}%
}

\hypersetup{pdftitle={Лабораторный практикум (9-ый семестр, 2021)}}

\begin{document}

\subsubsection*{Лабораторный практикум (10-ый семестр, 2022)}
\section*{Моделирование диска Галактики}
\subsubsection*{Руководитель: А. В. Веселова \hspace{2em} Выполнил: П. Л. Соболев}

\vspace{3em}

\subsection*{Задачи}

\begin{itemize}
  \setlength\itemsep{-0.1em}
  \item Создать (путем использования генератора случайных чисел) несколько каталогов пробных систем типа диска, таких что галактоосевое расстояние и расстояние от плоскости подчиняются экспоненциальному распределению.
\end{itemize}

\subsection*{Теория}

Для получения выборки по радиусу необходимо воспользоваться методом обратного преобразования: применить к случайным числам из равномерного распределения $ \mathrm{Unif}[0, 1] $ обратную функцию радиального распределения. Будем считать, что объекты диска распределены равномерно по всем направлениям. Тогда

\su
\begin{equation}
\begin{gathered}
  F_R(r) = P(R \leqslant r) = P(S \leqslant \pi r^2) = \int_{0}^{\pi r^2} f_S(s) \, ds; \\
  \textit{для каждого} \; x_0 \; \textit{из} \; U \sim \mathrm{Unif}[0, 1] \; \textit{имеем} \;\, y_0 = F^{-1}_R(x_0) \sim f_R(r).
\end{gathered}
\end{equation}

Плотности вероятности равномерного и экспоненциального распределений \\ имеют следующий вид:

\su
\begin{equation}
  f_{\mathrm{Unif}[a, b]}(x) = \begin{cases}
    \frac{1}{b - a} & a \leq x \leq b, \\
    0 & x < a \; \textit{или} \; x > b;
  \end{cases}
\end{equation}

\su
\begin{equation}
  f_{\mathrm{Exp}(\lambda)}(x) = \begin{cases}
    \lambda e^{-\lambda x} & x \geq 0, \\
    0 & x < 0.
  \end{cases}
\end{equation}

Тогда для радиальных распределений имеем

\su
\begin{equation}
\begin{gathered}
  F_{R, \, \mathrm{Unif}[0, R_0]}(r) = \frac{r^2}{R_0^2}, \; F_{R, \, \mathrm{Unif}[0, R_0]}^{-1}(x) = \sqrt{x} R_0^2, \; f_{R, \, \mathrm{Unif}[0, R_0]}(r) = \frac{2 r}{R_0^2}; \\
F_{R, \, \mathrm{Exp}(\lambda)}(r) = 1 - e^{-\pi \lambda r^2}, \; F_{R, \, \mathrm{Exp}(\lambda)}^{-1}(x) = \sqrt{-\frac{1}{\pi \lambda} \ln{(1 - x)}}, \\
f_{R, \, \mathrm{Exp}(\lambda)}(r) = 2 \pi \lambda r e^{-\pi \lambda r^2},
\end{gathered}
\end{equation}

где $ R_0 $ --- радиус диска, а $ \lambda = 1 / b $ --- обратный коэффициент масштаба. \npar

\subsection*{Реализация}

Каталоги и графики для визуализаций получены с помощью скрипта, написанного на языке программирования \href{https://julialang.org}{Julia}. Код расположен в GitLab репозитории \href{https://gitlab.com/paveloom-g/university/s10-2022/stellar-astronomy-laboratory-workshop}{Stellar Astronomy Laboratory Workshop S10-2022}. Для воспроизведения результатов следуй инструкциям в файле {\footnotesize \texttt{README.md}}. \npar

Для начала проверим генератор случайных чисел (\texttt{Xoshiro256++}) на тестовых распределениях (см. Рис. 1). Размер каталога здесь и далее полагается равным 1000 объектов.

\captionsetup{justification=centering}

\begin{figure}[h!]
  \centering
  \setlength{\imageheight}{4.65cm}
  \subgraphics{uniform/Histogram}
  \subgraphics{normal/Histogram}
  \subgraphics{exponential/Histogram}
  \subgraphics{laplace/Histogram}
  \caption{Гистограммы тестовых распределений \\ $ \mathrm{Unif}[1, 2] $, $ \mathrm{N}(1, \sqrt{2}) $, $ \mathrm{Exp}(2) $, $ \mathrm{Laplace}(0, 2) $} и соответствующие плотности вероятности
\end{figure}

Теперь создадим каталог для пробной системы диска с равномерным распределением ($ R_0 = 2.0 \; \text{кпк} $) в плоскости симметрии и экспоненциально распределенным ($ b = 0.5 \; \text{кпк} $) расстоянием от плоскости симметрии (см. Рис. 2--3).

\begin{figure}[h!]
  \centering
  \setlength{\imageheight}{4.07cm}
  \subgraphics{uniform_disk/2/Histogram R}
  \subgraphics{uniform_disk/2/Histogram θ}
  \subgraphics{uniform_disk/2/Histogram Z}
  \caption{Гистограммы выборок из распределений значений радиуса, угла и высоты и соответствующие плотности вероятности}
\end{figure}

\newpage

\begin{figure}[h!]
  \centering
  \setlength{\imageheight}{5.1cm}
  \subgraphics{uniform_disk/2/Heatmap Symmetry Plane}
  \subgraphics{uniform_disk/2/Heatmap Vertical Cut}
  \caption{Тепловые карты плоскости симметрии и \\ вертикального осевого сечения}
\end{figure}

Теперь положим галактоосевое расстояние также распределенным экспоненциально. В качестве параметров возьмем $ b_{r} = 2.6 \; \text{кпк} $ и $ b_{z} = 0.9 \; \text{кпк} $ (см. Bland-Hawthorn \& Gerhard (2016)). Создадим 3 таких каталога (см. Рис. 4--9).

\begin{figure}[h!]
  \centering
  \setlength{\imageheight}{4.07cm}
  \subgraphics{exponential_disk/1/Histogram R}
  \subgraphics{exponential_disk/1/Histogram θ}
  \subgraphics{exponential_disk/1/Histogram Z}
  \caption{Гистограммы выборок из распределений значений радиуса, угла и высоты и соответствующие плотности вероятности (первый каталог)}
\end{figure}

\begin{figure}[h!]
  \centering
  \setlength{\imageheight}{5.1cm}
  \subgraphics{exponential_disk/1/Heatmap Symmetry Plane}
  \subgraphics{exponential_disk/1/Heatmap Vertical Cut}
  \caption{Тепловые карты плоскости симметрии и \\ вертикального осевого сечения (первый каталог)}
\end{figure}

\newpage

\begin{figure}[h!]
  \centering
  \setlength{\imageheight}{4.07cm}
  \subgraphics{exponential_disk/2/Histogram R}
  \subgraphics{exponential_disk/2/Histogram θ}
  \subgraphics{exponential_disk/2/Histogram Z}
  \caption{Гистограммы выборок из распределений значений радиуса, угла и высоты и соответствующие плотности вероятности (второй каталог)}
\end{figure}

\begin{figure}[h!]
  \centering
  \setlength{\imageheight}{5.1cm}
  \subgraphics{exponential_disk/2/Heatmap Symmetry Plane}
  \subgraphics{exponential_disk/2/Heatmap Vertical Cut}
  \caption{Тепловые карты плоскости симметрии и \\ вертикального осевого сечения (второй каталог)}
\end{figure}

\begin{figure}[h!]
  \centering
  \setlength{\imageheight}{4.07cm}
  \subgraphics{exponential_disk/3/Histogram R}
  \subgraphics{exponential_disk/3/Histogram θ}
  \subgraphics{exponential_disk/3/Histogram Z}
  \caption{Гистограммы выборок из распределений значений радиуса, угла и высоты и соответствующие плотности вероятности (третий каталог)}
\end{figure}

\newpage

\begin{figure}[h!]
  \centering
  \setlength{\imageheight}{5.1cm}
  \subgraphics{exponential_disk/3/Heatmap Symmetry Plane}
  \subgraphics{exponential_disk/3/Heatmap Vertical Cut}
  \caption{Тепловые карты плоскости симметрии и \\ вертикального осевого сечения (третий каталог)}
\end{figure}

\end{document}
